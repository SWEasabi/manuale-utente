\chapter{Utente Gestore}

\section{Cosa può fare l'Utente Gestore}
L'Utente Gestore ha il potere di svolgere diverse attività all'interno del sistema. Di seguito, vengono elencate le principali funzionalità:

\subsection{Aprire Ticket}
L'Utente Gestore ha la possibilità di aprire nuovi ticket. 
I ticket rappresentano richieste o problemi segnalati all'\textit{Utente Manutentore}.
Ogni ticket viene gestito da un entità esterna.

\subsection{Visualizzare Ticket}
L'Utente Gestore può visualizzare l'elenco dei ticket esistenti. 

\subsection{Visualizzare le Entità: Area, Lampada e Sensori}
Il Gestore ha la possibilità di visualizzare le diverse entità del sistema, tra cui:
\begin{itemize}
\item \textbf{Area}: Rappresenta una zona geografica gestita dall'applicazione, dove possono essere collocate una o più lampade e sensori. 
\item \textbf{Lampada}: Indica le fonti luminose nel sistema. Il Gestore può visualizzare informazioni sullo stato delle lampade ed il loro alias.
\item \textbf{Sensori}: Rappresentano i sensori collocati in diverse aree del progetto. Il Gestore può visualizzare le informazioni dei sensori come Alias, posizione geografica e portata.
\end{itemize}

