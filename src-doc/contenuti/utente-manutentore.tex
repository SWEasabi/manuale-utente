\chapter{Utente Manutentore}

\section{Cosa può fare l'Utente Manutentore}
L'Utente Manutentore ha il potere di svolgere diverse attività all'interno del sistema. Di seguito, vengono elencate le principali funzionalità:

\subsection{Chiudere Ticket}
L'Utente Manutentore ha la possibilità di chiudere ticket. 
I ticket rappresentano richieste o problemi segnalati dall'\textit{Utente Gestore}.

\subsection{Visualizzare Ticket}
L'Utente Manutentore può visualizzare l'elenco dei ticket esistenti. 

\subsection{Inserimento/rimozione delle Entità: Area, Lampada e Sensori}
Il Manutentore ha la possibilità di modificare le diverse entità del sistema, tra cui:
\begin{itemize}
\item \textbf{Area}: Rappresenta una zona geografica gestita dall'applicazione, dove possono essere collocate una o più lampade e sensori. 
\item \textbf{Lampada}: Indica le fonti luminose nel sistema. Il Manutentore può visualizzare informazioni sullo stato delle lampade ed il loro alias.
\item \textbf{Sensori}: Rappresentano i sensori collocati in diverse aree del progetto.Il Manutentore può visualizzare le informazioni dei sensori come Alias, posizione geografica e portata.
\end{itemize}